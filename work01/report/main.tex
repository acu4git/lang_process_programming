\documentclass{jlreq}

\usepackage{bm}
\usepackage{fancyhdr}
\usepackage{float}
\usepackage{graphicx}

\pagestyle{fancy}
\fancyhf{}
\fancyhead[R]{\thepage}

\renewcommand\thesubsection{(\alph{subsection})}

\title{言語処理プログラミング 課題1}
\author{22122502 川口 栄宗}
\date{提出日: \today}

\begin{document}

\maketitle
\clearpage

\section{作成したプログラムの設計情報}

\subsection{全体構成}
モジュールの関係を図\ref{fig:module_map}に示す.

\begin{figure}[H]
  \centering
  \includegraphics[width=\textwidth]{assets/lpp01_module.png}
  \caption{モジュール構成図}
  \label{fig:module_map}
\end{figure}

\subsection{各モジュールごとの構成}
モジュール間で共通した変数を以下に示す.
\begin{table}[H]
  \centering
  \caption{モジュール間で共通して用いる変数}
  \begin{tabular}{|c|c|c|}
    \hline
    変数名 & 型    & 説明                                                    \\  \hline
    cbuf   & int   & ファイルから読み込んだ次の1文字の文字コードを格納する. \\ \hline
    fp     & FILE* & ファイルポインタ                                        \\ \hline
  \end{tabular}
\end{table}

\subsubsection{字句解析の初期化}
init\_scan関数内で初期化を行う.初期化処理では,mpplファイルの読み込み,ファイルポインタを用いた1文字の読み込み,
linenumへの0の代入,num\_attributeを-1にする処理を施す.

\subsubsection{主処理}
解析対象の字句に対してfgetc(fp)でファイルの文字を1文字ずつ読み込み,先頭の文字などで場合分けを行うことで
字句解析をscan関数内で行っている.scanが-1を返したときに,main関数内での解析処理は終わり,end\_scan関数を呼び出して主な処理は終了する.
エラーが発生した場合は戻り値としてS\_ERROR(-1)を返すとともに,error関数で標準エラー出力にエラーメッセージを表示して字句解析を終了させる.

\subsubsection{字句解析の終了}
end\_scan関数内で行う.ここでは,mpplファイルを閉じる処理のみが実行される.

\subsection{各関数の外部(入出力)使用}

\subsubsection{init\_scan}
\begin{description}
  \item[機能] スキャナーを初期化する
  \item[引数] ファイル名(char型へのポインタ)
  \item[戻り値] 成功の場合は0,ファイルオープンに失敗した場合は-1
\end{description}

\subsubsection{end\_scan}
\begin{description}
  \item[機能] ファイルを閉じて,スキャナーを終了する
  \item[引数] なし
  \item[戻り値] なし
\end{description}

\subsubsection{scan}
\begin{description}
  \item[機能] 字句解析を行う
  \item[引数] なし
  \item[戻り値] 字句解析に成功した場合は字句のコード(int型),コメントや改行を検出した場合は0,字句解析に失敗した場合はS\_ERROR(-1)
\end{description}

\subsubsection{push\_char}
\begin{description}
  \item[機能] 配列string\_attrに読み込んだ1文字を追加する
  \item[引数] char型の文字
  \item[戻り値] 成功の場合は0,string\_attrのサイズを超えた場合は-1
\end{description}

\subsubsection{pop\_char}
\begin{description}
  \item[機能] 配列string\_attrの末尾の文字を1つ取り出す
  \item[引数] なし
  \item[戻り値] 成功の場合は取り出した文字(int型),string\_attrから取り出せる文字がない場合は-1
\end{description}

\subsubsection{clear\_string\_attr}
\begin{description}
  \item[機能] string\_attrの中身を終端文字で埋めつくす(消去処理)
  \item[引数] なし
  \item[戻り値] なし
\end{description}

\subsubsection{is\_keyword}
\begin{description}
  \item[機能] 予約語かどうかを判定する
  \item[引数] 文字列(char型へのポインタ)
  \item[戻り値] 文字列が予約語の場合はトークン番号,そうでなければ-1
\end{description}

\subsubsection{get\_linenum}
\begin{description}
  \item[機能] 現在の行番号を取得する
  \item[引数] なし
  \item[戻り値] 行数(int型)
\end{description}

\subsubsection{skip\_blank}
\begin{description}
  \item[機能] スペースやタブなどの空白を読み飛ばす
  \item[引数] なし
  \item[戻り値] なし
\end{description}

\subsubsection{skip\_bracket\_comment}
\begin{description}
  \item[機能] 波括弧のコメント部分を読み飛ばす
  \item[引数] なし
  \item[戻り値] なし
\end{description}

\subsubsection{skip\_slash\_comment}
\begin{description}
  \item[機能] バックスラッシュを用いたコメント部分を読み飛ばす
  \item[引数] なし
  \item[戻り値] なし
\end{description}

\subsubsection{error}
\begin{description}
  \item[機能] エラーメッセージを標準エラー出力に表示する
  \item[引数] 文字列(char型へのポインタ)
  \item[戻り値] エラー番号(-1)を返す
\end{description}

\section{テスト情報}

\subsection{テストデータ}
\subsubsection{ブラックボックステスト}
ブラックボックステストには用意されていたテストデータを用いた.テストデータを以下に示す.
\begin{itemize}
  \item sample11.mpl
  \item sample011.mpl
  \item sample11p.mpl
  \item sample11pp.mpl
  \item sample12.mpl
  \item sample012.mpl
  \item sample12cr.mpl
  \item sample12crlf.mpl
  \item sample012eof.mpl
  \item sample12lf.mpl
  \item sample12lfcr.mpl
  \item sample012n.mpl
  \item sample012neof.mpl
  \item sample012s.mpl
  \item sample012seof.mpl
  \item sample13.mpl
  \item sample013.mpl
  \item sample14.mpl
  \item sample014.mpl
  \item sample014a.mpl
  \item sample014b.mpl
  \item sample14p.mpl
  \item sample15.mpl
  \item sample15a.mpl
  \item sample16.mpl
  \item sample17.mpl
  \item sample18.mpl
  \item sample19p.mpl
\end{itemize}

\subsection{テスト結果}
\subsection{テストデータの十分性}

\section{課題のスケジュールと実際の進捗状況}

\subsection{事前計画}
\subsection{実際の進捗状況}
\subsection{当初の事前計画と実際の進捗との差の原因}

\end{document}